\documentclass[12pt]{article}

\usepackage[a4paper, top=25mm, right=30mm, bottom=25mm, left=30mm]{geometry}
\usepackage[russian]{babel}
\usepackage{fontspec}
\usepackage{graphicx}
\usepackage[unicode]{hyperref}

\setmainfont{Times New Roman}
\graphicspath{{images/}}

\title{Lab4 Informatics title}
\author{Азат Сиразетдинов}

\begin{document}
	\thispagestyle{empty}
	\begin{center}
		Федеральное государственное автономное образовательное учреждение\\ 
		высшего образования\\
		«Национальный исследовательский университет ИТМО»\\
		\textit{Факультет Программной Инженерии и Компьютерной Техники}\\
	\end{center}
	\vspace{2cm}
	\begin{center}
		\large
		\textbf{Лабораторная работа № 6}\\
		по дисциплине информатика\\
		Работа с электронными таблицами\\
		Вариант № 17
	\end{center}
	\vspace{7cm}
	\begin{flushright}
		Выполнил:\\
		cтудент  группы P3116\\
		Сиразетдинов А. Н\\
		Преподаватель: \\
		Машина Е.А.\\
	\end{flushright}
	\vspace{6cm}
	\begin{center}
		г. Санкт-Петербург\\
		2022г.
	\end{center}
	\newpage
	
	
	
	\tableofcontents
	\newpage
	\section{Задание}
	A = 12893\\
	C = 13547\\
	X1 = 12893, 
	X2 = 13547,
	X3 = 26440, 
	X4 = 39987, 
	X5 = 654, 
	X6 = 25549, 
	X7 = -12893, 
	X8 = -13547, 
	X9 = -26440, 
	X10 = -39987,
	X11 = -654, 
	X12 = -25549.
	Область допустимых значений: от $-2^15$ до $2^15-1$
	\section{Вывод}
	В ходе работы я научился работать в системе \LaTeX и узнал интересные факты из журнала Квант.
\end{document}
