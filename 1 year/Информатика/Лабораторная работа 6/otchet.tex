\documentclass[12pt]{article}

\usepackage[a4paper, top=25mm, right=30mm, bottom=25mm, left=30mm]{geometry}
\usepackage[russian]{babel}
\usepackage{fontspec}
\usepackage{graphicx}
\usepackage[unicode]{hyperref}

\setmainfont{Times New Roman}
\graphicspath{{images/}}

\title{Lab4 Informatics title}
\author{Азат Сиразетдинов}

\begin{document}
	\thispagestyle{empty}
	\begin{center}
		Федеральное государственное автономное образовательное учреждение\\ 
		высшего образования\\
		«Национальный исследовательский университет ИТМО»\\
		\textit{Факультет Программной Инженерии и Компьютерной Техники}\\
	\end{center}
	\vspace{2cm}
	\begin{center}
		\large
		\textbf{Лабораторная работа № 6}\\
		по дисциплине информатика\\
		Работа с системой компьютерной вёрстки \LaTeX \\
		Вариант № 66
	\end{center}
	\vspace{7cm}
	\begin{flushright}
		Выполнил:\\
		cтудент  группы P3116\\
		Сиразетдинов А. Н\\
		Преподаватель: \\
		Машина Е.А.\\
	\end{flushright}
	\vspace{6cm}
	\begin{center}
		г. Санкт-Петербург\\
		2022г.
	\end{center}
	\newpage
	\tableofcontents
	\newpage
	\section{Задание}
	Год выпуска: 1976\\
	Выпуск: 6\\
	Страницы: 74, 32\\
	\begin{minipage}[h]{0.49\linewidth}
		\includegraphics[width=0.8\linewidth]{page1_img}
	\end{minipage}
	\hfill
	\begin{minipage}[h]{0.49\linewidth}
		\includegraphics[width=0.8\linewidth]{page2_img}
	\end{minipage}
	\newpage
	
	\section{Выполнение работы}
	\url{https://github.com/MakeCheerfulUpload/laboratornaya-rabota-5-Azat2202}
	\newpage
	
	\section{Вывод}
	В ходе работы я научился работать в системе \LaTeX и узнал интересные факты из журнала Квант.
\end{document}