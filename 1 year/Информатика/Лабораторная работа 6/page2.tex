
\begin{tabularx}{0.95\textwidth}{>{\hsize=.6\hsize\linewidth=\hsize}X
			>{\hsize=1.5\hsize\linewidth=\hsize}X}
 \textbf{M347. } \textit{Двое играют в такую игру. Первый загадывает два числа от 1 до 25, а второй должен их угадать. Он может назвать любые два числа от 1 до 25 и узнать у первого сколько из названных им чисел - 0, 1 или 2 - совпадают с загаданными. За какое минимальное число вопросов он сможет наверняка определить загаданные числа?}
 \begin{wraptable}{l}{1\linewidth}	
		\begin{center}
		\begin{tabular}{|c|c|}
			\hline
			Ответ & Загаданы \\
			& числа\\
			\hline
		 1 . 1 & 2i . 25\\
		 1 . 0 & 2i - 1 . 23\\
		 1 . 0 & 2i - 1 . 23 \\
		 0 . 1 & 2i  - 1 . 24\\
		 0 . 0 & 2i-1 . 25\\
			\hline
		\end{tabular}
		\caption{}
		\label{guessednumbers}
		\end{center}
	\end{wraptable} & Многие читатели успешно справились с определением загаданных чисел за 14 вопросов. Покажем, что всегда можно определить загаданные числа не более чем за 13 вопросов. 
	Называя пары $ (1, 2), (3, 4), \dots ,(21, 22) $, мы используем 11 вопросов; при этом возможны следующие 4 случая:
	\begin{enumerate}[label=\alph*)]
		\item после какого-то вопроса получен ответ "2";
		\item на все вопросы получены ответы "0d";
		\item на какие-то два вопроса - i -й и j -й - получены ответы "1";
		\item только на один i -й вопрос получен ответ "1", на остальные вопросы - "0" (невнимательное рассмотрение этого случая многих заставило считать, что нельзя гарантировать определение загаданных чисел из 13 вопросов).
	\end{enumerate}

	Укажем дальнейшие действия отгадывающего в каждом из этих случаев. 
	\begin{enumerate}[label=\alph*)]
		\item После ответа "2" загаданные числа определены.
		\item Загаданы два числа из чисел 23,24,25. Задаем вопрос (23,24). Если ответ "2", то эти числа и загаданы, если ответ "1", то вопросом (23,22) определим, какое из чисел - 23 или 23 - загадано наряду с числом 25.
		\item Числа в i  паре (2i, 23), (2i , 23) при всех возможных ответах определяют загаданные числа. В самом деле, ответ "2" на первый или второй вопрос не требует пояснений. Для других комбинаций ответов на эти два вопроса мы сообщаем загаданные числа (легко проверяется, что другого мнения о том, какие числа загаданы, не может быть)-см. таблицу \ref{guessednumbers}
	\end{enumerate}

	Итак, мы показали, что за 13 вопросов всегда можно определить загаданные числа; естественно, как следует из решения, иногда хватает и меньшего количества вопросов. 
	
	Для завершения решения докажем, что нельзя гарантировать определение загаданных чисел за 12 вопросов. После 11 вопросов все ответы могут быть "0"; при этом всегда существуют три числа, не включенные в вопросы. Если двенадцатый вопрос не содержит ни одно из этих трех чисел, то ответ "0" позволит любым двум из них быть загаданными. Если же в двенаддцатый вопрос входит одно или два из этих трех чисел, то после ответа "1" также нельзя однозначно указать загаданные числа. 
	
	\begin{flushright}
		\textit{Ю. Лысов}
	\end{flushright}
\end{tabularx}