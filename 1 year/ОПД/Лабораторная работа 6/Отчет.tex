\documentclass[12pt]{article}
\usepackage[a4paper, left=5mm, right=5mm, top=5mm, bottom=5mm]{geometry}
%\usepackage[a4paper, top=15mm, right=10mm, bottom=10mm, left=10mm]{geometry}
\usepackage[russian]{babel}
\usepackage{fontspec}
\usepackage{graphicx}
\usepackage[unicode]{hyperref}
\usepackage{enumitem}
\usepackage{wrapfig}
\usepackage{tabularx}
\usepackage{amssymb}
\usepackage{gensymb}
\usepackage{amsmath}
\usepackage{blindtext}
\usepackage{float}
\usepackage{multicol}
\usepackage{latexsym}
%\usepackage[font={bf}, name={Рис. }, justification=justified]{caption}
\usepackage{caption}
\usepackage{subcaption}
\usepackage{listings}
\usepackage{xcolor}
\usepackage{breqn}
\usepackage{pdfpages}

\setmainfont{Times New Roman}
\righthyphenmin=2 % правильные переносы
%\graphicspath{{images/}} % путь к картинкам

% Стиль кода для 5 задания
\lstdefinestyle{mystyle}{
	backgroundcolor=\color{backcolour},   
	commentstyle=\color{codegreen},
	keywordstyle=\color{magenta},
	numberstyle=\tiny\color{codegray},
	stringstyle=\color{codepurple},
	basicstyle=\ttfamily\footnotesize,
	breakatwhitespace=false,         
	breaklines=true,                 
	captionpos=b,                    
	keepspaces=true,                 
	numbers=none,                    
	numbersep=5pt,                  
	showspaces=false,                
	showstringspaces=false,
	showtabs=false,                  
	tabsize=2
}
\lstset{style=mystyle}

\title{Lab6 OPD}
\author{Азат Сиразетдинов}

\begin{document}
	\thispagestyle{empty}
	\begin{center}
		Федеральное государственное автономное образовательное учреждение\\ 
		высшего образования\\
		«Национальный исследовательский университет ИТМО»\\
		\textit{Факультет Программной Инженерии и Компьютерной Техники}\\
	\end{center}
	\vspace{2cm}
	\begin{center}
		\large
		\textbf{Лабораторная работа № 6}\\
		по дисциплине ОПД\\
		Обмен по прерыванию\\
		Вариант № 1323
	\end{center}
	\vspace{7cm}
	\begin{flushright}
		Выполнил:\\
		cтудент  группы P3116\\
		Сиразетдинов А. Н\\
		Преподаватель: \\
		Афанасьев Д. Б.\\
	\end{flushright}
	\vspace{6cm}
	\begin{center}
		г. Санкт-Петербург\\
		2023г.
	\end{center}
	\newpage
	
	\tableofcontents
	
	\newpage
	\section{Задание}
	
	По выданному преподавателем варианту разработать и исследовать работу комплекса программ обмена данными в режиме прерывания программы. Основная программа должна изменять содержимое заданной ячейки памяти (Х), которое должно быть представлено как знаковое число. Область допустимых значений изменения Х должна быть ограничена заданной функцией F(X) и конструктивными особенностями регистра данных ВУ (8-ми битное знаковое представление). Программа обработки прерывания должна выводить на ВУ модифицированное значение Х в соответствии с вариантом задания, а также игнорировать все необрабатываемые прерывания.
	\begin{itemize}
		\item Основная программа должна уменьшать на 3 содержимое X (ячейки памяти с адресом 03916) в цикле.
		\item Обработчик прерывания должен по нажатию кнопки готовности ВУ-3 осуществлять вывод результата вычисления функции F(X)=3X+6 на данное ВУ, a по нажатию кнопки готовности ВУ-2 изменить знак содержимого РД данного ВУ и записать в Х
		\item Если Х оказывается вне ОДЗ при выполнении любой операции по его изменению, то необходимо в Х записать максимальное по ОДЗ число.
	\end{itemize}
	\newpage
	
	\section{Описание программы}
	\subsection{Назначение программы}
	Программа в цикое декрементирует значение ячейки $ 039 $. По нажатию кнопки готовности ВУ-2 изменяет знак содержимого РД данного ВУ и записывает в ячейку $ 039 $. По нажатию кнопки готовности ВУ-2 осуществляет вывод результата вычисления функции F(X)=3X+6 на данное ВУ. В случае выхода за ОДЗ записывает в $ 039 $ максимальное по ОДЗ число
	
	\subsection{Область представления}
	X - знаковое целое 16-разрядное число, значащими являются 8 младших разрядов
	
	\subsection{Область допустимых значений}
	
	$ X \in \left[-2C;28\right] $
	
	\subsection{Расположение программы и данных}
	Программа располагается в ячейках с 50 по 65\\
	Вектора прерываний располагаются в ячейках с 0 по F\\
	Обработчики прерываний располагаются в ячейках с 10 по 1E\\
	Исходные данные располагаются в ячейках:
	\begin{itemize}
		\item 39 - Х
	\end{itemize}
	Константы данные располагаются в ячейках:
	\begin{itemize}
		\item 40 - Х\_MAX со значением 28
		\item 41 - Х\_MIN со значением -2C
	\end{itemize}
	
	\section{Исходный код программы}
	\includegraphics[width=2\linewidth]{colored_code.pdf}
	\newpage
	
	\section{Методика проверки}
	\begin{enumerate}[]
		\item Загрузить комплекс программ в память базовой ЭВМ
        \item Изменить значение точки останова по адресу 11 на HLT
		\item Изменить значение точки останова по адресу 17 на HLT
        \item Изменить значение точки останова по адресу 19 на HLT
		\item Изменить значение точки останова по адресу 1B на HLT
  	\item Изменить значение точки останова по адресу 1E на HLT
   	\item Изменить значение точки останова по адресу 24 на HLT
		\item Запустить программу в автоматическом режиме с адреса 50
		\item Открыть "КВУ-2"
		\item Установить значение 1000000 (вне ОДЗ)
		\item Установить "Готовность КВУ-2"
		\item Дождаться останова
  	\item Записать значение аккумулятора
        \item Продолжить исполнение программы
        \item Дождаться останова
		\item Проверить что значение AC равно 0000 0000 0010 1000 (X\_MAX)
		\item Продолжить исполнение программы
        \item Дождаться останова
        \item Сопоставить значение аккумулятора и записанного значения
        \item Продолжить исполнение программы
		\item Открыть "КВУ-2"
		\item Установить значение 0000 0000 0001 1000
		\item Установить "Готовность КВУ-2"
		\item Дождаться останова
  	\item Записать значение аккумулятора
        \item Продолжить исполнение программы
        \item Дождаться останова
		\item Проверить что значение AC равно 1111 1111 1110 1000
		\item Продолжить исполнение программы
        \item Дождаться останова
        \item Сопоставить значение аккумулятора и записанного значения
        \item Продолжить исполнение программы
		\item Открыть "КВУ-3"
		\item Установить "Готовность КВУ-3"
		\item Дождаться останова
        \item Записать значение аккумулятора
        \item Продолжить исполнение программы
        \item Дождаться останова
		\item Записать значение AC как переменную х
		\item Продолжить исполнение программы
        \item Дождаться останова
        \item Сопоставить значение аккумулятора с записанным ранее
		\item Сопоставить значение DR ВУ-3 и ожидаемым значением формулы 3x+6
        \item Продолжить исполнение программы
	 \end{enumerate}
 	\newpage
 
 	\section{Вывод}
 	В процессе выполнения лабораторной работы я узнал про работу с внешними устройствами по прерыванию и про блокировку для предоставление атомарности операции. Была написана программа реализующую работу с прерываниями ВУ-2 и ВУ-3 и разработана методика проверки. 
\end{document}







