\documentclass[12pt]{article}

\usepackage[a4paper, top=25mm, right=30mm, bottom=25mm, left=30mm]{geometry}
\usepackage[russian]{babel}
\usepackage{fontspec}
\usepackage{graphicx}
\usepackage[unicode]{hyperref}

\setmainfont{Times New Roman}
\graphicspath{{images/}}

\title{Lab4 Informatics title}
\author{Азат Сиразетдинов}

\begin{document}
    \thispagestyle{empty}
    \begin{center}
        Федеральное государственное автономное образовательное учреждение\\
        высшего образования\\
        «Национальный исследовательский университет ИТМО»\\
        \textit{Факультет Программной Инженерии и Компьютерной Техники}\\
    \end{center}
    \vspace{2cm}
    \begin{center}
        \large
        \textbf{Лабораторная работа № 4}\\
        по дисциплине программирование\\
        Исключения в Java \\
        Вариант № 9865.6
    \end{center}
    \vspace{7cm}
    \begin{flushright}
        Выполнил:\\
        cтудент  группы P3116\\
        Сиразетдинов А. Н\\
        Преподаватель: \\
        Кустарев Иван\\
    \end{flushright}
    \vspace{6cm}
    \begin{center}
        г. Санкт-Петербург\\
        2022г.
    \end{center}
    \newpage
    \tableofcontents
    \newpage
    \section{Задание}
    И он схватил мумию в охапку и понес в прихожую.
    И вот наконец в щель почтового ящика кто-то просунул проволоку.
    Собственно говоря, Малыш и Карлсон этого не увидели, потому что
    в тамбуре было темно, хоть глаз выколи, а услышали: раздалось полязгивание
    и скрежет, так что сомнений быть уже не могло -- вот они, долгожданные Филле и
    Рулле! Все это время Малыш и Карлсон просидели на корточках под
    круглым столиком     в прихожей и ждали. Так прошло не меньше
    часа. Малыш даже задремал. Но он разом проснулся, когда в ящике
    что-то заскрежетало. Ой, вот сейчас все начнется!
    С него мигом слетел всякий сон, ему было так страшно,
    что по спине забегали мурашки. Карлсон решил его ободрить.
    Подумать только, что с помощью простой проволочки можно
    так легко сдвинуть "собачку"! Потом дверь осторожно приоткрыли,
    и кто-то проскользнул в нее, кто-то был здесь, в тамбуре!
    У Малыша перехватило дыхание -- это и в самом де ле было невероятно.
    Послышались шепот и тихие шаги... И вдруг раздался грохот -- о,
    что за грохот! -- и два приглушенных вскрика. И только тогда
    Карлсон под столом зажег свой фонарик и тут же его снова потушил,
    но на краткий миг луч света упал на наводящую ужас, устрашающую,
    смертоносную мумию, которая стояла, прислоненная к стене, и в зловещей
    улыбке скалила зубы -- зубы дяди Юлиуса. И снова крики, на этот
    раз более громкие. Все дальнейшее произошло как-то одновременно,
    и Малыш не смог ни в чем разобраться. Он слышал, как распахнулись
    двери, -- это выскочили из своих комнат дядя Юлиус и фрекен Бок,
    и тут же он услышал чьи-то шаги в тамбуре. Карлсон потянул Мамочку
    к себе за поводок, который он надел ей на шею, и она с глухим стуком
    упала на пол. Потом он услышал, как фрекен Бок несколько раз повернула
    выключатель, чтобы зажечь свет в прихожей, но он не зажигался,
    потому что Карлсон выкрутил все пробки на предохранительном щитке
    на кухне. "Проказничать лучше в темноте", -- сказал он. И вот фрекен
    Бок и дядя Юлиус беспомощно стояли, не зная, как осветить прихожую.
    Но фрекен Бок стала его уверять, что это наверняка гром, она не может
    ошибиться. Собственно говоря, он сказал "шкашошные шушештва", потому
    что он стал вдруг шепелявить. "Он ведь остался без зубов", -- догадался
    Малыш, но тут же об этом забыл. Он мог думать только о Филле и
    Рулле. Где они? Убежали? Он не слышал, чтобы хлопнула входная дверь.
    Вероятнее всего, они стоят где-то в тамбуре, притаившись в темноте,
    может, спрятались за пальто, висящие на вешалке? О, до чего ж страшно!
    Малыш придвинулся как можно ближе к Карлсону.
    \newpage
    \section{Исходный код}
    \url{ https://github.com/Azat2202/Prog_lab4}
    \newpage
    \section{Вывод программы}
    \url{ https://github.com/Azat2202/Prog_lab4/out.txt}
    \newpage
    \section{Диаграмма классов}
    \url{ https://github.com/Azat2202/Prog_lab4/ClassesUML.png}
    \newpage
    \section{Вывод}
    В процессе работы над лабораторной работой я научился работать с исключениями, сделал сам собственные классы исключений.
    Узнал про типы классов и использовал их.
\end{document}